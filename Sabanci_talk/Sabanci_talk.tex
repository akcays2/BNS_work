\documentclass[xcolor=dvipsnames,handout,t]{beamer}
% use handout in document class to stop transitions

\mode<presentation>
{
  \usetheme{Madrid}%{Madrid}
  % or ..AnnArbor, Boadilla, CambridgeUS, Copenhagen, default, Frankfurt, Goettingen, Hannover
  %       Montpellier, PaloAlto, Rochester, Szeged,

  \setbeamercovered{invisible}
  % or whatever (possibly just delete it)
}
\usepackage{color,empheq}
\usepackage{subfigure}
\usepackage{caption}
\usepackage{subcaption}
\usepackage[english]{babel}
\usepackage{xmpmulti,animate}
\usecolortheme{seahorse} %this makes the colors less strong.

\usepackage{times}
% Or whatever. Note that the encoding and the font should match. If T1
% does not look nice, try deleting the line with the fontenc.
\usepackage{graphicx}
%\usepackage{tcolorbox}
\usepackage{tikz,lipsum,lmodern}
\usepackage[most]{tcolorbox}
\usepackage{amsfonts,amssymb,amsmath,mathtools}
      % use Times fonts if available on your TeX system
\usepackage{epsfig}
\usepackage{hyperref}
\usepackage{changepage}
\usepackage[utf8]{inputenc}
\usepackage[T1]{fontenc}
%\usepackage{animate,movie15,media9}
%\usepackage{multimedia}

\newcommand{\todo}[1]{\textcolor{orange}{\texttt{TODO: #1}}} 
\newcommand{\red}[1]{\textcolor{red}{#1}} 
\newcommand{\bl}[1]{\textcolor{blue}{#1}} 


\usetikzlibrary{shapes}

\title[] % (optional, use only with long paper titles)


%\subtitle
%{Presentation Subtitle} % (optional)

\title[Forecasting GRBs w/ GWs] % (optional, use only with long paper titles)
{Forecasting Gamma-ray Bursts with Gravitational Waves}

%\subtitle
%{Presentation Subtitle} % (optional)

\author[Sarp Akcay]{Sarp Akcay\inst{1}\inst{2} }
\institute[FSU Jena - UCD] % (optional, but mostly needed)
{
  \inst{1}%
  FSU Jena %and University College Dublin
  \inst{2}%
  University College Dublin
  }


%\vspace{-3cm}
%\titlegraphic{
%  \vspace{1.75cm}  \includegraphics[width=1.0cm]{figs/ucd.png}%\hspace*{1cm}~\includegraphics[height=1cm]{figs/missing_inch.jpg} \hspace*{1cm}  \includegraphics[width=1.0cm]{figs/ucd.png}
%}
\date[Sabanci University]{14 November 2018}

\subject{Talks}

\usefonttheme[onlymath]{serif}
\setbeamerfont{frametitle}{size=\huge}
\setbeamercolor{frametitle}{fg=Black,bg=White}

\newcommand{\Mag}{\textcolor{magenta}}
\newcommand{\Red}{\textcolor{red}}
\newcommand{\Blue}{\textcolor{blue}}
\renewcommand{\c}{\cos}
\renewcommand{\t}{\theta}
\renewcommand{\b}{\bar}
\newcommand{\f}{\frac}
\newcommand{\bt}{\beta}
%\newcommand{\s}{\sin}
\newcommand{\ph}{\phi}
\newcommand{\g}{\gamma}
\newcommand{\nn}{\nonumber}
\newcommand{\la}{\lambda}
\newcommand{\al}{\alpha}
\newcommand{\La}{\Lambda}
\newcommand{\el}{\ell}
\newcommand{\h}{\hat}
\newcommand{\mrm}{\mathrm}
\newcommand{\ord}{\mathcal{O}}
\newcommand{\F}{\mathcal{F}}
\newcommand{\be}{\begin{equation}}
\newcommand{\ee}{\end{equation}}
\newcommand{\ba}{\begin{eqnarray}}
\newcommand{\ea}{\end{eqnarray}}
\newcommand{\bi}{\begin{itemize}}
\newcommand{\ei}{\end{itemize}}
\newcommand{\bef}{\begin{frame}}
\newcommand{\ef}{\end{frame}}
%\newcommand{\h}{\bar{h}}
\newcommand{\bs}{\begin{small}}
\newcommand{\es}{\end{small}}
\newcommand{\parallelsum}{\mathbin{\!/\mkern-5mu/\!}}
\newcommand{\Lim}[1]{\raisebox{0.5ex}{\scalebox{0.8}{$\displaystyle \lim_{#1}\;$}}}
\newcommand*\circled[1]{\tikz[baseline=(char.base)]{\node[shape=circle,draw,inner sep=2pt] (char) {#1};}}
%------------------ for coloured boxes in math modes --------------------
% Syntax: \colorboxed[<color model>]{<color specification>}{<math formula>}
\newcommand*{\colorboxed}{}
\def\colorboxed#1#{%
  \colorboxedAux{#1}%
}
\newcommand*{\colorboxedAux}[3]{%
  % #1: optional argument for color model
  % #2: color specification
  % #3: formula
  \begingroup
    \colorlet{cb@saved}{.}%
    \color#1{#2}%
    \boxed{%
      \color{cb@saved}%
      #3%
    }%
  \endgroup
}
%%-------------------------------------------------------------------------

% tikz boxes
\newcommand\TBox[3][]{%
  \tikz\node[draw,ultra thick,red,text width=#2,align=left,#1] {#3};}

\setbeamertemplate{frametitle}[default][center]



\begin{document}
%{
%\usebackgroundtemplate{\includegraphics[width=1\paperwidth,height=\paperheight]{../Figures/ET_strains_redshifted_v2.pdf}}
\begin{frame}
 \titlepage
\end{frame}
%}



\begin{frame}{Motivation: Multi-messenger Astronomy}
\begin{itemize}
 \item \emph{Multi}: two fundamentally different waves from common events.
 \item[]\quad Gravitational waves (GWs): ``hearing''  the universe.
 \item[]\quad Electromagnetic waves (EWs): ``seeing'' the universe.
 \item GW170817-GRB170817A-AT2017gfo: the ``event'' \\
 \quad inspiral + merger of two neutron stars (GWs) \\
 \ \ = short-hard gamma-ray burst + kilonova (EWs)
 \item[] Broad-band in gravitational waves: 10Hz - 2000Hz
 \item[] Ultra broad-band in electromagnetic waves: gamma rays to radio.
 \item A personal motivation, a simple question \\
{\small ``What can we offer to the astronomy community with future GW detections?''}
 \end{itemize}


 \end{frame}

 
\begin{frame}{Gravitational waves: Einstein 1916}
  \begin{center}
\begin{tikzpicture}
            \node[anchor=south west,inner sep=0] at (0,0) {\includegraphics[height=3.5cm]{figs/Einstein1916.png}};
            %\draw<1>[red,ultra thick,rounded corners] (1.6,1) rectangle (\textheight-1cm,5);
           \uncover<3->{\draw<3->[red,thick] (1.7,0.04) rectangle (3.1,0.27);}
        \end{tikzpicture}
        \\
         \uncover<2->{Perturbations of spacetime with speed $=c$, sourced by accelerating masses.}
         \end{center}
          \begin{itemize}
 \item  \uncover<3->{Spacetime metric $g_{\mu\nu} = \eta_{\mu\nu} + h_{\mu\nu}$, \quad with $| h_{\mu\nu} | \ll 1$.}
 \item  \uncover<4->{Insert into $G_{\mu\nu} = 0$. Keep only $\mathcal{O}(h)$. Pick a gauge.} %(e.g. Lorenz).}
  \uncover<5->{\[ \Rightarrow\quad \Box \bar{h}_{\mu\nu}= 0, \quad \text{wave equation!} \]}
  \item \uncover<6->{\alert{Plane-waves:} $ \bar{h}_{\mu\nu} = \Re\left[A_{\mu\nu}\, \text{e}^{i k_\mu x^\mu}\right] =\Re\left[A_{\mu\nu}\, \text{e}^{-i\omega (t-z/c )}\right]$.}
 \end{itemize}
\end{frame}

\begin{frame}{Gravitational waves: d.o.f.'s}
$A_{\mu\nu} $ is the polarization tensor.
\begin{itemize}
 \uncover<2->{\item $\{ \bar{h}_{\mu\nu}, A_{\mu\nu} \}$: $4\times4$, symmetric \uncover<3->{$\hspace{1.5cm}\Rightarrow \f{4\times 5}{2}\ \ \ = 10$ d.o.f.}}
 \uncover<4->{\item Gauge condition: $\nabla_\mu \bar{h}^{\mu\nu} =  A_{\mu\nu} k^\nu= 0$ \uncover<5->{$\Rightarrow10-4 = \, 6$ d.o.f.}}
 \uncover<6->{\item Residual gauge freedom \hspace{2.25cm} \uncover<7->{$\Rightarrow 6-4 \ \ = \alert{2}$ \alert{d.o.f.}}}
 %\uncover<7->{\vspace{2mm}\begin{center}$ \bar{h}'_{\mu\nu} = \bar{h}_{\mu\nu} - \xi_{\mu,\nu} - \xi_{\nu,\mu} + \eta_{\mu\nu} \nabla_\alpha \xi^\alpha . $ \end{center}\vspace{2mm}}
\end{itemize}
\uncover<7->{\quad \ Consistent with $\pm 2$ helicities of a \alert{massless} spin 2 boson.\\}
\uncover<8->{$\Rightarrow$ \alert{2} POLARIZATIONS (\alert{transverse}): \uncover<10->{$\begin{array}{l}A_{xx}=-A_{yy} = h_\oplus  \\ A_{xy}=\ \ A_{yx} = h_\otimes  \\ \end{array}$}}

{\begin{center} \includegraphics[height=2.79875cm]{figs/white_box.png}\end{center}}
\end{frame}
 
 

\begin{frame}{Gravitational waves: d.o.f.'s}
%\begin{center}
%\animategraphics[loop,controls,width=3cm]{12}{GWs_frame-}{0}{29} 
%\end{center}
$A_{\mu\nu} $ is the polarization tensor.
\begin{itemize}
{\item $\{ \bar{h}_{\mu\nu}, A_{\mu\nu} \}$: $4\times4$, symmetric {$\hspace{1.5cm}\Rightarrow \f{4\times 5}{2}\ \ \ = 10$ d.o.f.}}
 {\item Gauge condition: $\nabla_\mu \bar{h}^{\mu\nu} =  A_{\mu\nu} k^\nu= 0${ $\Rightarrow10-4 = \,6$ d.o.f.}}
 {\item Residual gauge freedom \hspace{2.25cm}{ $\Rightarrow 6-4 \ \ = \alert{2}$ \alert{d.o.f.}}}
 %{\vspace{2mm}\begin{center}$ \bar{h}'_{\mu\nu} = \bar{h}_{\mu\nu} - \xi_{\mu,\nu} - \xi_{\nu,\mu} + \eta_{\mu\nu} \nabla_\alpha \xi^\alpha . $ \end{center}\vspace{2mm}}
\end{itemize}
{\quad \ Consistent with $\pm 2$ helicities of a \alert{massless} spin 2 boson.\\}
{$\Rightarrow$ \alert{2} POLARIZATIONS (\alert{transverse}): {$\begin{array}{l}A_{xx}=-A_{yy} = h_\oplus  \\ A_{xy}=\ \ A_{yx} = h_\otimes  \\ \end{array}$}}
{\begin{center} \animategraphics[loop,controls,height=2.4cm]{12}{GWs_plus-}{0}{20} \hspace{1cm} \animategraphics[loop,controls,height=2.4cm]{12}{GWs_cross-}{0}{20}\end{center}}
\end{frame}

\begin{frame}{Making gravitational waves}{non-spherical, accelerated motion}
 \alert{Inspirals}, GRBs, bumps on neutron stars, supernovae, the Big Bang. \\% (BICEP 2).\\
 \uncover<2->{Gravitational-wave luminosity: \uncover<3->{\alert{quadrupole} radiation}}
 \uncover<3->{\begin{center} $ \boxed{\dot{E}=L_\text{GW} = \f{G}{c^5} \langle (\partial_t^3 \bar{Q}_{ij})^2 \rangle}$\\\vspace{2mm}}
 \uncover<4->{\begin{small}$ \quad \bar{Q}_{ij} = Q_{ij}-\frac{1}{3}\delta_{ij}Q^k_k, \quad Q^{ij} = \int \rho(t,{\mathbf x})x^i x^j d^3 x $ \end{small} \end{center}}
 \uncover<5->{{\bf Binary systems:} 2 points masses in circular orbit. }
 
\end{frame}

\begin{frame}{Making gravitational waves}{non-spherical, accelerated motion}
 \alert{Inspirals}, GRBs, bumps on neutron stars, supernovae, the Big Bang. \\%  (BICEP 2).\\
 Gravitational-wave luminosity: \alert{quadrupole} radiation
 \begin{center} $ \boxed{\dot{E}=L_\text{GW} = \f{G}{c^5} \langle (\partial_t^3 \bar{Q}_{ij})^2 \rangle}$\\ \vspace{2mm}
 \begin{small}$ \quad \bar{Q}_{ij} = Q_{ij}-\frac{1}{3}\delta_{ij}Q^k_k, \quad Q^{ij} = \int \rho(t,{\mathbf x})x^i x^j d^3 x $ \end{small} \end{center}
 {\bf Binary systems:} 2 points masses in circular orbit. \todo{Make $\dot{E}$s consistent}
 \begin{tikzpicture}[overlay,remember picture]
\onslide<5->\node (img0 )[anchor=center,scale=1,opacity=1] at ([shift={(-3.8cm,-2.8cm)}]current page.center) {\animategraphics[loop,controls,height=2.4cm]{12}{Circ_Orb-}{0}{20}};
\onslide<2->\node (t1 )[anchor=center,scale=1,opacity=1] at ([shift={(2cm,-1.3cm)}]current page.center) {$M=m_1+m_2,\ x(t) = R \cos\Omega t, \ y(t) = R \sin\Omega t .$};
\onslide<3->\node (t2 )[anchor=center,scale=1,opacity=1] at ([shift={(3cm,-2.2cm)}]current page.center) {$ Q^{ij} =  2 M\, x^i x^j, \quad \boxed{\dot{E} = \f{128}{5}\f{G}{c^5} \Omega^6 M^2 R^4}$.};
\onslide<4->\node (t3 )[anchor=center,scale=1,opacity=1] at ([shift={(2.5cm,-3.2cm)}]current page.center) {Enhancement due to \red{$e$} (Peters \& Mathews).};
\onslide<5->\node (t3 )[anchor=center,scale=1,opacity=1] at ([shift={(2cm,-4cm)}]current page.center) {$ \dot{E} = \textcolor{red}{\f{\left(1 + \f{73}{24}e^2+ \f{37}{96} e^4 \right)}{(1-e^2)^{7/2}}} \f{32}{5}\f{G}{c^5}\f{M m_1^2 m_2^2}{a^5}.$};
\end{tikzpicture}
\end{frame}








 \begin{frame}{Binary Neutron Star Inspirals}
  
 \end{frame}





 \begin{frame}{GW Detectors: Interferometers}
  Geodesic deviation
 \end{frame}





















\end{document}


